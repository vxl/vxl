  \documentclass[11pt]{article}
  \parskip1mm\parindent0mm\advance\textheight5cm\advance\textwidth4cm\advance\oddsidemargin-2cm\advance\topmargin-2cm
  \def\TargetJr{{T\kern-0.2em arget\kern-0.3em\lower.5ex\hbox{\Large J}\kern-0.13em r}}
  \def\by#1.{{\normalsize\sf\it(#1)}}\relax
  \edef\@{@}\edef\<{<}\edef\|{|}
  \catcode64=\active\def@#1@{{\em#1\/}}
  \catcode124=\active\def|#1|{{\tt#1}}
  \catcode60=\active\def<#1>{\hspace*{-3mm}{\bf#1}}
  \catcode42=\active\def*#1*{\hfil\penalty-8000\section{#1}}
  \catcode95=\active\let_\_
  \begin{document}

  \centerline{\bf\large
Minutes of the \TargetJr{} meeting at Orlando, December 9\&10, 1999.
  }


* Participants: *

  \begin{tabbing}
Manchester: ~ \= Dave Cooper, Alan Beveridge \\
GE/CRD:       \> Joe Mundy, Richard Hartley, Don Hamilton, Peter Tu \\
Oxford:       \> Andrew Zisserman, Andrew Fitzgibbon, David Capel, Geoff Cross,
                 Karen McGaul, Fred Schaffalitzky \\
Leuven:       \> Peter Vanroose, Joris Schouteden \\
Zurich:       \> Manuel Oetiker, Stephan Scholze, Andreas Turina \\
  \end{tabbing}


* Overview and GE demo \by Joe Mundy. *

<GE concerns:>

-- \TargetJr{} fragmentation

-- Repository fragility

-- Too infrequent coordination

-- Departure of Bill Hoffman

<Recent developments at GE:>

-- Improved junction closure of Canny edge detection through @region analysis@
(class |IntensityFace|).  Allows planar intensity fit.  Note that regions are
subpixel.

-- boolean operations (intersection, \dots) on solids, implemented using
BSPTree.  Useful for (e.g.) refining 3D shape from contours in different
views.


* Oxford demo \by Andrew Fitzgibbon. *

demo of new `vgui'.


* Leuven overview \by Peter Vanroose. *

overview of new functionality, and of research projects using \TargetJr.


* Reorganisation of \TargetJr{} packages \by Andrew Fitzgibbon. *

Ideally, we want a software environment which is easy to learn/read.\\
@keywords:@ |consistency|; |power| (algos+interface);
|lightweight|; |compatibility| (e.g.~MMX).

<Ways of achieving this:> rewrite (the model used when Target junior was
created from Target), springclean (used for \TargetJr{} version 3).
Rewrite has the advantage that above goals are achievable.  The
disadvantage of any springclean is that bugs are reintroduced.

-- Proposed libraries in the `cleanland' |v$x$l| suite:
|vcl| (ISO compatibility level), |vbl| (basics layer), |vsl| (spatial objects),
|vnl| (numerics), |vil| (imaging), |vgl| (GUI). \\
An example application (xcv) should be included.

<Did the following experiment:>
-- Can xcv be rewritten from scratch (but `cleanly') in 1 day by 4 people?
Participants: Andrew Fitzgibbon, Karen McGaul, Fred Schaffalitzky, David Cooper.
4 hours at the whiteboard, setting up guiderules and structure, then
4 hours of programming (including grabbing existing \TargetJr{} code).
At that point, numerics demos run, basics library contains 20 ``.h'' files

-- In order to obtain such a result, @clean coding rules@ are needed.
(Note that it will be difficult to decide on such rules, as tastes differ\ldots)
AWF: It may be difficult, but there will be agreement.
Appendix 3 contains an initial list, as discussed at the Oxford/Manchester meeting.

<Discussion:>

-- There will be problems in re-writing algorithms: now depending on (say)
Topology, must be rewritten in an independent way.  This could cost several days
per algorithm.

-- How to reconcile different @paradigms@, and possibly perform paradigm
change (from ``old but functional'' code to ``new, clean'' code)~? \\
Examples: embedding dimension splitup (|IUPoint_2d| -- |IUPoint_3d|);
          |CoolList| node sharing vs.~STL;

-- To avoid that non-clean code enters `cleanland', some @people@ have to
function as whatchdog, and perform the actual move.

-- How can we assure that `smart, exotic' behaviour (like large image support,
hacks needed for a specific platform, \ldots) is not destroyed by this move?

-- Have to avoid code duplication!  I.e., move implementation, and leave a
`pointer' to it in the old place.
Gradually, the old \TargetJr{} will be re-implemented in terms of the new
design.

<Summary:>

-- Rationalise the basic functionality in \TargetJr{}, forming a
``reptilian brain'' area, around which `cleanland' can be built.

-- Gradually re-implement existing algorithms and structures in `oldland',
taking care of maintaining its functionality.

-- The resulting @clean@ system alone should be capable of doing non-trivial
demos:  we will define 5 ``killer demos'' (like VanDuc Canny), see further.

-- @Timing:@ design meeting by the end of January (3 to 5 days, 5 to 8 people,
possibly in Oxford), which results in a clean reptilian brain core system.
The complete cleanland system must be ready before ECCV and CVPR (June 2000).


* \TargetJr{} and the IUE \by Joe Mundy. *

<The current situation:>

DARPA is stopping (after 25 years) its financial support for image processing
research, so the IUE has no money any more.  AAI will no longer be able to
continue its IUE support.

<How to proceed:>

either (A) IUE goes its own way (if there is still an active IUE community), \\
or     (B) we incorporate the IUE bits in our restructured \TargetJr{}
           (in much the same way as we do with the \TargetJr{} bits). \\
Good points in the IUE are e.g. its neat @structure@, its transform graph
network (local coordinate systems, and transforms between neighbours,
cf. robotics), templated images, its documentation system, and the algorithms
embedded in the |Tasks-IUE| package.

We will ask the IUE community (|iue-users\@aai.com|) which path (A or B) they
prefer.  No explicit choice in favour of A will imply B.

* Makefiles \by Richard Hartley. *

<Current situation:>
in order to build a new application, one has to spend several hours
finding out what the |IULIBS| variable should look like.

The point is that I don't (want to) know that libraries which I need, are
themselves using a (possibly changing) set of other libraries.

<Proposal:>
It is possible to setup the makefiles so that |IULIBS| is automatically
determined, using a pre-generated global |iu_library_table| and an
|IU_LibDependencies| file in every library directory.

Your (application) library only has to specify |IMMEDIATE_LIBS|, include
the auto-generated |LDLIBS.mk| and then set |IULIBS| equal to
\verb|$(STDLIBS)|.  %$
|USES| is no longer used to find libraries, only include files.
Actually, |USES| can be dropped completely.

<Discussion:>

-- There are some concerns, like the danger of over-crosslink libraries,
of makefiles becoming less readable, and of increased unwanted
library/package interdependency because the makefile author does not have to
specify |USES| any more, so has no control over unwanted dependencies.

-- However, the advantages are more important, so we go for it.
The objections can easily be overcome by appropriate changes to |rules.mk|,
and by still requiring |USES|, but only issuing a warning when something
outside of that list is being used.


* STL (versus COOL) and portability status *

-- Shall we start replacing |CoolList| etc.?  This is very difficult because of
the different semantics (be it in maybe only 1\% of the used places).

-- The `cleanland' code will exclusively use STL.


* Smart pointers \by Richard Hartley. *

Using smart pointers avoids memory leaks without any extra effort.

However, the current implementation (in Fresco's CORBA) is {\bf bad}:
not the smart pointer @constructor@ should increment the refcount,
the @assignment@ to something of type |\ldots_ref| should do.

In the new implementation (|GeneralUtility/Basics/smart_ptr.h|),
every |RefCntTimeStampMixin| can be made into a smart pointer.
(Actually, every class which has methods |Protect()| and |UnProtect()|.)

Note that it is still up to the user to actually @use@ smart pointers.

Example:
  {\obeylines\tt
typedef IUE_smart_ptr\<Image> Image_ref; // this is in Image_ref.h
Image_ref ptr = new Image(...);
  }

%%% RICHARD, COULD YOU INSERT YOUR CODE EXAMPLES HERE, PLEASE?

We will include this in the ``reptilian brain'' design.


* Video processing in \TargetJr \by Richard Hartley. *

GE has a new Video package, which operates on either AVI files (NT only) or
on Image sequences.  A demo is shown, where we see the video playing, with
spatial object overlay (Harris corners). \\
The package will be released soon, after some necessary cleanup.

@Internals:@ classes |VideoMovie| (= list of clips), |VideoTake| (= header info:
file + number of frames + list of frames), |VideoClip| (= ptr to |VideoTake|),
|VideoFrame| (= ptr to codec), |VideoCodec| (sort of ImageInterface + framenr.),
|VideoAsImage| (derived class from |Image|, also contains current frame nr.).

For the GUI: all overlay planes (one for each frame) are populated in advance,
before starting the video display.  There is only one image plane, which is
updated with the different frames during display of the video.

<Discussion:>

-- Patent issue: do we support MPEG?  (We cannot distribute freeware MPEG
readers/writers with a \TargetJr{} release without paying royalties).

-- Do we need 4-D (time-aware) spatial objects?  Note that a 4D object is more
than just a list of 3D objects, one for each frame: a continuous time notion
will e.g. allow to specify object @speed@ (which is necessary for e.g.
predictive image coding).


* The next release *

Should be ready just before ECCV and CVPR, i.e., June 1, 2000.  On ECCV a
tutorial day is scheduled devoted to TargetJr/IUE (Peter Vanroose).  Joe
will try to organize a similar event at CVPR.

Should contain 5 ``killer demos'', which should be completely written in
`cleanland' code.  Suggestions are:

-- VanDuc Canny, Harris corner detection, line fitting.

-- 3D geometry (boolean ops, BSPTree, Cameras, VRML I/O).

-- geometry GUI interaction, in the style of |xfig|.
E.g.: edit conic parameters, draw bitangent~\ldots\
Also: plane overlay + update (timestamp); RTree selection speed; extrusion.

-- MultiView: find point correspondences, compute fundamental matrix, display
epipolar geometry.

-- a 10 to 20 line code example, showing how easy it is to program \TargetJr.

-- other suggestions are welcome (e.g.: Video; splines/snakes; \ldots)


<Timing:>

  \begin{tabbing}
February 1: ~ \= design meeting, first version of core `reptilian brain'. \\
March 1:      \> end of pass 1, including the 5 demos.                    \\
April 1:      \> end of pass 2, including fully functional demos.         \\
May 1:        \> documentation (incl. example code) ready.                \\
June 1:       \> CD release, possibly including NT binaries.
  \end{tabbing}

* A new name ? *

-- Several suggestions are made, but nothing really appealing came up.
Acronyms versus full names (of, e.g., historical persons, from literature,
opera, \ldots)

-- New suggestions are welcome.  The final decision is for June 1, 2000.


* GUI issues --- the Oxford/Manchester VGUI design *

-- David Cooper sketches GUI history at Manchester: \\
never liked Fresco; tcl(/tk) looked promising, but does not scale well to
large GUIs; now prefer Qt (but: need to work out object display).

<Introduction> \by Andrew Fitzgibbon.

-- a GUI is a necessary I/O interface to a package like \TargetJr.

-- but try to be as independent as possible from the graphics package used.

-- @2 design choices:@

- VGUI deals with geometry, not widgets/buttons\ldots
~(note: this was one of Parmesan's mistakes.)\\
$\Rightarrow$ able to use MFC or Qt or GTK or glut or glX or Motif or \ldots \\
VGUI codes the geometry, the graphical package does the rendering and widgets.

- Assume fast OpenGL is available (with Mesa as a fallback). \\
(Note: OpenGL has a pure 3D interface: you give it 3D directives: absolute
coordinates and camera position ($4\times4$ matrix).)
Assuming that only OpenGL is supported greatly simplifies the design.
(Should maybe support Direct3D in the future?)

<VGUI design:>

A @tableau@ is a screen rectangle (``canvas/context/subwindow''), which
receives events and draws GL.  It has methods |draw()| and |handle(event e)|.
(Actually, |draw()| is an event, not a method.)

An application consists of several tableaux.  Some have already been
predefined, like an image drawer, an image zoomer, an `acetate' (which just
draws its children one after another), a point overlay, an image raster \ldots

Example of an application: video with point overlay.  Consists of $2+3n$
tableaux: a top ``image zoom'' tableau, an ``image deck'' with $n$
``acetate'' children, each having an image drawer and point overlay as
children.

<Discussion:>

-- some technical issues (``does it do 16-bit images or colormapped images?'')
will have to be answered in the coming days.

-- how can we provide backward compatibility with Parmesan?  nobody seems to be
willing to write a Fresco implementation, although this is perfectly possible
but probably time consuming.

-- There is no general camera support in GL; how to use a non-linear camera,
like radial distortion?
\\ Answer: you will need to do the rendering manually, line by line, so
should write an appropriate tableau (VGUI_nonlinear_camera_tableau for
example).

<Case study: GTK implementation> \by Karen McGaul.

%%% KAREN, PLEASE FILL IN DETAILS.

<Status:> almost ready for release, which is planned for January 8, 2000. \\
there are plans for a netscape plugin. \\
MFC: has to be done.


* \TargetJr{} and VTK \by Joris Schouteden. *

-- goal: GUI must be able to do 3D editing/picking, which vtk provides.
         should be easily extensible $\Rightarrow$ plugin system.

-- New |ImportVtk| package provides a \TargetJr{} to vtk interface.
vtk is mainly used for visualisation (on top of OpenGL).

-- GUI independent C++ @manipulators@ have been set up for interaction.
They receive events from GUI, which cause calling one of the functions
|finish()|, |cancel()| or |temporary()|.

A demo is shown (uses Qt as GUI): \\
image can be rotated; corner detector $\Rightarrow$ point overlay.
3D points can be selected.


* VTK and \TargetJr{} \by Peter Tu. *

The @Visualisation ToolKit@ is open source, $ > 500 $ C++ classes.
See |http://www.kitware.com|.

Rendering primitives: point, line, polygon, triangular strip, volume, \ldots

Easy to use (e.g.) \TargetJr{} Numerics within a vtk application.

Currently not very strong on 2D, not recommended for 2D work.


* Java advanced imaging (JAI) and Java3D \by Joe Mundy. *

These are new Java libraries to support image operations;
still too slow to be used.

VanDuc has ported \TargetJr's Topology to Jave3D.


* GUI: discussion and way forward *

Integrating the different ideas (VGUI, vtk, Qt, Fresco):

-- VGUI is an interesting concept;
it must be shown that it can interoperate with vtk (maybe modify the VGUI
design slightly?)

-- One solution : make a vtk VGUI tableau, or make a VGUI vtk renderer \\
-- Other solution: (partially) separate design, use VGUI for 2D, vtk for 3D. \\
-- Or: make a vtk actor which delegates to a tableau (which may be better
   because vtk is more general than vgui; e.g. it does not assume OpenGL).

-- We have to make sure that the new GUI can do the current sophisticated things
like displaying 16-bit images, colourmapped images, parts of very large images \ldots

-- We should not make the same errors as when we went from InterViews to Fresco.
which are?

-- Inform |iue-users| in January (say) so that they know as soon as is sensible
that there is going to be an alternative to Fresco/Pamesan $-$ this will also
enable feedback from them.

* Repository (CVS) issues *

-- Apparently there have been ``bad'' CVS commits in September.
How to avoid this in the future?

-- Idea: give every active group a time slot (1 week?) during which no other
commits can happen.  This allows them to prepare the commit and make sure
everything compiles OK before doing so.

-- Maybe this is only necessary for ``large'' or ``deep'' commits.  The current
policy at GE (daily (small) commits by several researchers) works perfectly.
Infrequent CVS updates and commits have the important disadvantage that
conflicts occur more frequent, in which case (and also in the case of compile
errors) the person who was responsible for the `bad' commit cannot remember
why he made that change, which makes it very difficult to find the correct fix.

-- We must make sure that the central (Leuven) repository compiles OK at all
times.  We will set up a dashboard (accessible through http) where this can
be verified.  There are about 20 platform/compiler combinations which we
actively support (Linux/Solaris/IRIX/WinNT; gcc 2.7, 2.91, 2.95, VC 5.0, 6.0,
CC-n32; complete list in appendix), so we need 20 computers which do a nightly
build and make their logfile available for inclusion on the dashboard.


* Identify @core@ \TargetJr{} -- avoid Balkanization \by Joe Mundy. *

An example: there are 11 different line intersection algorithm implementations
in \TargetJr, not counting the IUE ones: 5 in |ImplicitLine|, 2 in |CompGeom|,
3 in |MultiView|, 1 in |Polyhedra|.

-- @Proposal:@  create a @Standard Geometry Library@ (SGL),
with ultra lightweight classes, both 2D and 3D, both Euclidean and projective.

-- @Issues:@ which representation? which interface for (e.g.) intersection? \\
which types of objects? (e.g., point, line, triangle? spline? box?)
infinite lines or line segments? \\
Tolerances: user-settable or default? used in |operator==()| or not?

-- Discussion: do we really need separating Euclidean and projective?
A 3D point is an abstract concept, should be able to behave as 3-vector
(Euclidean) or homogeneous 4-vector (projective).


* Documentation *

-- We want to use |perceps| instead of |gendoc|, and possibly re-define the
structured comments.

-- Zurich proposes to set up a web ``HowTo'' server page, which will be kept
up-to-date, containing code snippets and the possibility for anybody to add
examples.

-- The present example code should be kept up-to-date.  This could be made
sure automatically, by also running these examples (just like the tests)
during building, and verifying the output.

-- We need a better mailing list setup, so that users should not hesitate
to submit questions.

-- Stefan Scholze is willing to write a FAQ list (without answers ;-)


* ToDo list after this meeting * %%% PLEASE MAKE ADDITIONS !!

Contact |iue-users\@aai.com| to find out if they want to continue with IUE
on their own (Joe)

Organize VXL meeting in late Jan/early February (AWF)

Set up a HowTo server (Stefan)

Set up a CVS-repo compile status dashboard (Peter V)

Prepare for the big changes.  (Some ideas in appendix.)

Work out how to integrate different GUI ideas (Joris, AWF)

Release VGUI on January 8, 2000 (AWF)

\newpage


APPENDIX 1 --- Supported platforms, and who's responsible for nightly build:

  \begin{tabbing}

platform \>	compiler	 \>	organization	\\
Linux 2.x \>	gcc 2.95.2	 \>	Oxford	\\
Solaris 2.5 \>	gcc 2.7.2	 \>		\\
Solaris 2.5 \>	gcc 2.95.2	 \>		\\
Solaris 2.6 \>	gcc 2.7.2	 \>		\\
Solaris 2.6 \>	EGCS 1.1.x	 \>		\\
Solaris 2.6 \>	gcc 2.95.2	 \>	Leuven	\\
Alpha 4 \>	gcc 2.95.2	 \>		\\
HP-UX \>	gcc 2.95.2	 \>	Leuven	\\
WinNT \>	VC++ 5.0	 \>		\\
WinNT \>	VC++ 6.0	 \>	GE	\\

  \end{tabbing}

APPENDIX 2 --- Prepare for the big changes

The following initial steps can already be taken. \\
Report on this will help making the correct decisions in February:

-- CoolListP $\rightarrow$ CoolList etc. \\
-- |reset()|, |next()|, |value()| $\rightarrow$ |CoolList_iterator|, in
those `perlable' situations where a |for| loop only uses |value()|, not (e.g.)
|remove()| or |find()|. \\
-- CoolList $\rightarrow$ STL list, etc., with provisional automatic conversion
to allow gradual substitution.

\vskip1cm

APPENDIX 3 --- Coding conventions

\def\vxl{|v$x$l}
A subset of rules:
\begin{itemize}
\item All identifiers are lowercase with underscores separating words.

{\em Rationale:}
This choice is preferred as MixedCase identifiers are poisoned by Visual
C++ headers.


\item All class names (and file names) start with the
|vcl_| |vbl_| |vnl_| etc.~prefixes.

{\em Rationale:} Avoid conflict with other libraries.  If conflicts do
occur, adherence to the prefix rules means that names are more likely to be
easily changed automatically.

{\em Rejected alternatives:} (1) Use of the C++ namespace feature.  We do
not wish to depend on nearly-ready C++ functionality, as we have in the
past (e.g.\ usage of STL in the early days of the IUE).  (2) Use of
\verb|#define| to simulate namespaces.  Experience with InterViews and
Fresco shows that that way lies madness.


\item The file \vxl_$y$.h| defines one or more globally visible identifiers
(class names, functions, cpp macros).
All globally visible identifiers defined in |vxl_y.h| begin with |vxl_y|.

{\em Rationale:} Make it easy to locate the definitions of
identifiers. e.g.\ on seeing the symbol |vbl_rtree_iterator|, one must look
in library |vbl|, file |vbl_rtr|$\langle${\sc tab}$\rangle$

{\em Rejected alternatives:}
Stricter rules are possible, for example the file |vxl_y.h| defines only the
identifier |vxl_y|.  These will not be obeyed, so it is better to propose
rules that can be.


\item Each library contains the file \vxl_dll.h|.

{\em Rationale:} standard practice, this formalizes the naming.


\item Each library contains the file \vxl_fwd.h|, which forward-declares
all global identifiers in the library.

{\em Rationale:} Naming is by analogy with the ISO C++ \verb|<iosfwd>|
header.  One file per library means a larger recompilation burden whenever
a new class is added, but in normal use, new classes are added rarely.

{\em Rejected alternatives:} One file per \vxl| header, so that \vxl_$y$.h|
has a corresponding \vxl_$y$_fwd.h|.  This was considered too great a
burden on the user, so that they would be tempted to simply write |class
vbl_generic_image;| rather than \verb|#include <vbl/vbl_generic_image_fwd.h>|.


\end{itemize}


\end{document}
